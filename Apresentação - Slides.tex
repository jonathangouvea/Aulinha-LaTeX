\documentclass{beamer}
\usetheme{Pittsburgh}
\usecolortheme{whale}
\usepackage[utf8]{inputenc}
\usepackage[brazilian]{babel}

\definecolor{gray}{rgb}{0.5,0.5,0.5}

\usepackage{listings}
\lstset{
   aboveskip=3mm,
   belowskip=3mm,
   showstringspaces=false,
   columns=flexible,
   basicstyle={\small\ttfamily},
   numbers=left,
   language=[LaTeX]{TeX},
   numberstyle=\tiny\color{gray},
   breaklines=true,
   breakatwhitespace=true
   tabsize=3
   }


\title{\LaTeX}
\subtitle{Introdução}
\author{Jonathan Gouvea}
\date{2019}

\begin{document}

\begin{frame}
    \titlepage
\end{frame}

\begin{frame}{Agenda}
    \tableofcontents
\end{frame}

\section{Iniciando}

\begin{frame}{O que é \LaTeX}
    \begin{itemize}
        \item Sistema de preparação de documentos
        \item Utiliza tags de marcação para formatar o documento
        \item Documentos com aparência verdadeiramente profissional
        \item Focar no conteúdo e não na aparência
    \end{itemize}
\end{frame}

\begin{frame}{Material}
    \begin{itemize}
        \item Existem diversos compiladores
        \item Usaremos o Overleaf
    \end{itemize}
    
    \begin{figure}
    \includegraphics[scale=0.2]{ov.png}
    \caption{Logo do Overleaf}
    \end{figure}
\end{frame}

\section{Estrutura}
\begin{frame}[fragile]{Iniciando}
    \Large Estrutura básica
    \begin{lstlisting}
\documentclass[12pt,a4paper,notitlepage]{report}
\usepackage[utf8]{inputenc}

\title{Exemplo}
\author{Jonathan}
\date{2019}

\begin{document}
    \maketitle
\end{document}
    \end{lstlisting}
\end{frame}

\begin{frame}{Documentos}
    \begin{columns}
    \begin{column}{0.5\textwidth}
    \centering Tipos de documentos
    \begin{itemize}
        \item \texttt{article}
        \item \texttt{minimal}
        \item \texttt{report}
        \item \texttt{book}
    \end{itemize}
    \end{column}
    \begin{column}{0.5\textwidth}
    \centering Opções dos documentos
    \begin{itemize}
        \item \texttt{10pt, 11pt, 12pt}
        \item \texttt{a4paper, letterpaper}
        \item \texttt{titlepage, notitlepage}
        \item \texttt{onecolumn, twocolumn}
        \item \texttt{twoside, oneside}
        \item \texttt{onecolumn, twocolumn}
    \end{itemize}
    \end{column}
    \end{columns}
\end{frame}

\begin{frame}[fragile]{Tags principais}
    \begin{columns}
    \column{0.5\textwidth}\\
    Pular linha\\
    \textbf{Negrito}\\
    \textit{Itálico}\\
    \textsc{Small Caps}\\
    \texttt{Typewriter}\\
    \underline{Underline}\\
    \column{0.5\textwidth}
        \begin{lstlisting}
\\ ou \newline
\textbf{Negrito}
\textit{Itálico}
\textsc{Small Caps}
\texttt{Typewriter}
\underline{Underline}
        \end{lstlisting}
    \end{columns}    
\end{frame}

\begin{frame}{Títulos, Capítulos, Seções}
    \begin{itemize}
        \item \textsc{chapter}
        \item \textsc{section}
        \item \textsc{subsection}
        \item \textsc{subsubsection}
    \end{itemize}
\end{frame}

\begin{frame}{Tamanhos, Listagens}
    \begin{description}
        \item [Sizes]
            \begin{itemize}
                \item SMALL
                \item LARGE
                \item Large
                \item Footnotesize
                \item huge
            \end{itemize}
            
        \item [Listagens]
            \begin{itemize}
                \item description
                \item itemize
                \item enumerate
            \end{itemize}
    \end{description}
\end{frame}

\begin{frame}{Cross Reference}
    Diferentes maneiras de se referir a outra parte do texto
    \begin{itemize}
        \item label
        \item ref
        \item pageref
    \end{itemize}
    
    footnote
\end{frame}



\end{document}

