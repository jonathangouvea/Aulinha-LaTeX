\documentclass[a4paper, 12pt, oneside]{article}
\usepackage[utf8]{inputenc}

\usepackage{graphicx}

\usepackage[brazilian]{babel}
\usepackage{tgpagella}
\usepackage{indentfirst}

\usepackage[all]{nowidow}

\usepackage{enumitem}

%Tirar margem das listas
\setlist[itemize]{leftmargin=*}
\setlist[enumerate]{leftmargin=*}
\setlist[description]{leftmargin=*}

\usepackage{tocloft}

\usepackage{dblfloatfix}
\usepackage{float}

\usepackage{natbib}

\usepackage{xcolor}
\definecolor{dkgreen}{rgb}{0,0.6,0}
\definecolor{gray}{rgb}{0.5,0.5,0.5}
\definecolor{mauve}{rgb}{0.58,0,0.82}

\usepackage{listings}
\lstset{frame=tb,
   language=C, %C#
   aboveskip=3mm,
   belowskip=3mm,
   showstringspaces=false,
   columns=flexible,
   basicstyle={\small\ttfamily},
   numbers=left,
   numberstyle=\tiny\color{gray},
   keywordstyle=\color{blue},
   commentstyle=\color{dkgreen},
   stringstyle=\color{mauve},
   breaklines=true,
   breakatwhitespace=true
   tabsize=3
   }

\title{Teste}
\author{Jonathan Gouvea da Silva}
\date{April 2019}

\begin{document}

\maketitle

\tableofcontents
\newpage
\listoffigures
\clearpage

\section{Introdução}

Bom dia, meu nome é pipipi popopo.

Novo parágrafo. \\
Sem outro parágrafo. \newline
\newpage
Outra linha \label{marcador_outra_linha}

\textbf{Negrito} não negrito \textbf{e mais negrito} e menos.

\textit{Itálico}

\texttt{Typewrite}

\textsc{Small caps}

\underline{under}


\section{Pipipi}

Uma nova seção nasceu.

    \subsection{Popopo}
        Uma subseção.
        
    \subsection{Outra}
        Outra
        
    \subsection{Mais uma}
        \subsubsection{Outra}
            .
            
\newpage

\section{Tamanhos}

{\huge gigantesco oi}

{\LARGE gigantesco oi}

{\large gigantesco oi}

{\small gigantesco oi}

{\footnotesize gigantesco oi}

\begin{itemize}
    \item Item número 1
    \item numero 2
    \item 3
    \item outro
    
    \begin{itemize}
        \item Mais uma
        \item lista
    \end{itemize}
    
    Voltei para a lista
    
    \begin{itemize}
        \item Oi
        \begin{itemize}
            \item Oi 
        \end{itemize}
    \end{itemize}
    
\end{itemize}

\begin{enumerate}
    \item Oi
    \item Tchau
    \begin{enumerate}
        \item Mais uma
        \item Outra
        \begin{itemize}
            \item Oi
        \end{itemize}
    \end{enumerate}
\end{enumerate}

\footnote{Isso é um footnote}

\begin{description}
    \item[Olá] é uma palavra
    \item[Peixe] é uma comida
\end{description}

Como visto na página \pageref{marcador_outra_linha}, pipipi popopó.

\centering Texto

Bj\"ork

\begin{figure}[H]
    \centering
    \includegraphics[width=0.4\textwidth]{B.jpg}
    \caption{Um belo guaxinim}
    \label{fig:guaxinim1}
\end{figure}

\begin{figure}[!H]
    \centering
    \includegraphics[width=0.8\textwidth]{g.jpg}
    \caption{Um outro guaxinim}
    \label{fig:guaxinim2}
\end{figure}

Mais texto

\[ a + b = x + 2\]

Na equação \( c * 5 - 8 = -78\) vemos

$a*b = c $

\[ \frac{x}{5} \]

\[ \integral \]

\begin{lstlisting}
    //Código que faz parte do artigo para iniciantes em www.alexandregama.wordpress.com

#include<stdio.h>
#include<stdlib.h>

int main() {
   //Declaração das variáveis para guardar os valores
   int dividendo;
   int divisor;
   
   //Imprime mensagem para a inserção do primeiro valor
   printf("Digite o valor do dividendo:");
   //Guarda o valor digitado pelo usuário na variável dividendo
   scanf("%d", &dividendo);
   
   //Imprime mensagem para a inserção do segundo valor
   printf("\nDigite o valor do divisor:");
   //Guarda o valor digitado pelo usuário na variável divisor
   scanf("%d", &divisor);   
   
   //Verifica se o valor do divisor é igual a zero
   if (divisor == 0) {
      //Imprime o valor -1 caso o divisor seja zero
      printf("-1\n");
   }
   //Verifica se o valor do cálculo da divisão é negativo
   else if ((dividendo / divisor < 0)) {
      //Imprime o valor 0 caso o resultdo da divisão seja negativo
      printf("Valor encontrado: 0\n");
   }
   else {
      //Como o divisor não é zero e o cálculo não é negativo, imprime o resultado da divisão
      printf("Valor calculado: %d \n", (dividendo / divisor));
   }
   
   system("PAUSE");
}
\end{lstlisting}

\nocite{livro}

\bibliographystyle{plain}
\bibliography{referencia}

\end{document}

